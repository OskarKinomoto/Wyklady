\documentclass[a4paper,10pt]{article}
\usepackage[utf8]{inputenc}
\usepackage{polski}

%opening
\title{Wstęp do Fizyki Układów Złożonych}
\author{}

\begin{document}

\maketitle

\section{2016-10-04}
prof. d hab. Robert Kosiński. 

FUZ wyrasta z magnetyzmu.

\subsection{Wymagania formalne}
\begin{itemize}
 \item Zaliczenie na podstawie 2 kolokwiów (2-3 tematy opisowe bądź "problemiki" do rozwiązania)
 \item Na każdym kolokwium można 10 pkt.
 \item Zaliczenie suma min 10 pkt.
\end{itemize}

\subsection{Działy}
\begin{itemize}
 \item Zjawiska dynamiczne w sieciach złożonych
 \item Zastosowania
 \item Sztuczne sieci neuronowe
 \item Sieci neuronowe
\end{itemize}

\subsection{Literatura}
\begin{itemize}
 \item Sztuczne Sieci Neuronowe - R.A.Kosiński
 \item Complex Systems - T.R.Bossomaier Cambridge 2000
 \item Complexity - R.Badii A.Politi Cambridge 1997
\end{itemize}


\subsection{Układy złożone}
\paragraph{} Bardzo różne układy złożone opisuje się podobnymi równaniami.

\subsubsection{Przykłady układów złożonych}
\begin{itemize}
\item Typowym układem złożonym jest siec neuronowa człowieka ($10^{11}$ neuronów w mózgu człowieka, każdy neuron średnio 10k połączeń z innymi).
\item Społeczność ludzka (połączenia interpersonalne, hierarchia połączeń (rodzina, sąsiedzi itd.)) - sieć połączeń interpersonalnych - formowanie opinii, przenoszenie się chorób (epidemie)
\item Kwestie ekonomiczne, finansowe - przepływ kapitału, towarów
\item Granulaty - ziarna, piasek, gleba - lawiny ziemne/śnieżne - samo organizująca się krytyczność
\item Automaty komórkowe - Wykazują cechy życia
\end{itemize}

\subsubsection{Definicja}
% kolowium 1 lub 2!!
\paragraph{} Najczęściej są to układy składające się ze znacznej liczby oddziałujących elementów, które mogą być rozmaitej natury np układy biologiczne (sieć neuronowa), w tym układy jednostek o różnym 
poziomie inteligencji (społeczeństwa mrówek), układy elementów z przyrody nieożywionej (np gleby, sieci hydrologiczne), układy stworzone przez człowieka (np. systemy ekonomiczne, Internet). 
Do układów złożonych należą  też pewne matematyczne równania i odwzorowania (np automaty komórkowe, układy generujące obiekty charakteryzujące się \textit{pattern formation}).

\paragraph{} Są to układy o złożonej dynamice, zawsze nieliniowej, Mimo tej różnorodności, w prawach ewolucji układów złożonych obowiązują pewne uniwersalne prawa (np. prawa skalowania).

\paragraph{} Dokładny opis ilościowy takich układów najczęściej nie jest możliwy (nieliniowość !) zasadniczą rolę w ich opisie ogrywają obliczenia numeryczne i metoda symulacji komputerowych. 
Stąd gwałtowny rozwój badań nad takimi układami w ostatnich dekadach wywołany rozwojem technik komputerowych.

\subsection{Program}
\begin{enumerate}
\item Układy złożone w przyrodzie i ich badania,
\item Automaty komórkowe
\item Sztuczne życie
\item Samo organizująca się krytyczność
\item Inteligencja rozproszona
\item Sieci złożone - Sztuczne układy społeczne
\item Modelowanie wybranych zjawisk zachodzących w internecie
\item Zjawiska formowania wzorców
\item Sztuczne sieci neuronowe
\item Szkła spinowe (roztwór złota z domieszką żelaza)
\item Podsumowanie zjawisk charakterystycznych dla układów złożonych. Perspektywy rozwoju.
\end{enumerate}

\subsection{Przykłady}
\begin{itemize}
 \item Sieć neuronowa człowieka
 \item Sieć ethernetowa pomiędzy stronami
 \item Mapa interakcji białek drożdży
 \item Epidemie
 \begin{itemize}
  \item 1520 – 50\% of Aztecs died because of a smallpox epidemic
  \item 1919 – Hiszpanka 20 mln ludzi
  \item AIDS
  \item SARS
  \item Bioterrorist attacks
 \end{itemize}

\end{itemize}


\subsection{}
\paragraph{} Jest próg szczepień powyżej którego epidemie konkretnego patogenu nie rozprzestrzeniają się - następuje samo tłumienie
\paragraph{} \textit{pattern formation} - fraktale - płuca, liście, (quazi)periodyczne desenie, fale na oceanie (obserwowane przez satelity) - wykrywanie statków przez złamanie  patter formation
\paragraph{} Badania przesiewowe w celu automatyzacji wykrywania chorób, w wyniku zaburzeń pattern formation.
\paragraph{} pattern formation - w muszlach, mechanizm złozóny z 2 rodzajów centrów pigmentacji + silnych przesunięć fazowych między centrami
\end{document}
