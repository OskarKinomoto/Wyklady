\documentclass[a4paper,10pt]{article}
\usepackage[utf8]{inputenc}
\usepackage{polski}
\usepackage[utf8]{inputenc}
\usepackage[unicode]{hyperref}
\usepackage{amssymb}
\usepackage{xifthen}
\usepackage[fleqn]{amsmath}
\usepackage{todonotes}
\usepackage{graphicx}
\usepackage{fullpage}
\usepackage{amsmath}
\usepackage{hyperref}

\newcommand*{\hham}{\hat{\mathcal{H}}}

%opening
\title{Komputerowe Metody Symulacji}
\author{Oskar Świtalski}

\begin{document}

\maketitle

\section{Wykład I} %2016-10-04
\paragraph{Prowadzący} dr inż. Krzysztof Zberecki
\paragraph{Adres e-mail prowadzącego} \href{mailto:zberecki@if.pw.edu.pl}{zberecki@if.pw.edu.pl}
\paragraph{Strona z zadaniami} \url{http://if.pw.edu.pl/~zberecki/KMS/download.html}

\subsection{Program}
\begin{itemize}
 \item Metody deterministyczne, dynamika 
 \item Metody stochastyczne
 \item Podstawy chemii kwantowej w użyciu do  wł. mikro/nano układów (fizyki jądrowej/zderzenia ciężkich jonów)
 \item Omówienie algorytmów obliczeń z zbieżnych dziedzin
\end{itemize}


\subsection{Literatura}
\begin{itemize}
\item Podstawy symulacji komputerowych w fizyce – Hcerman
\item Chemia kwantowa – Kołos
\item Computential Physics – Koonin 
\item Computential Physics – Thijsen
\end{itemize}

\subsection{Zaliczenie}

\begin{itemize}
 \item Laboratorium – 2 programy:
 \begin{itemize}
  \item Mechanika molekularna
  \item Rozwiązanie układu Schrödingera od czasu
 \end{itemize}
 \item Kolokwium
\end{itemize}

\subsection{Metody symulacji komputerowych}
\begin{itemize}
 \item model $\rightarrow$ obserwable
 \item hamiltonian
 \item obserwable po zespole 
 \begin{gather*}
  \langle A \rangle = Z^{-1} \int_\Omega A(\vec{x}) f \big( \hham(\vec{x}) \big) dx \\
  \text{gdzie} \\
  Z = \int_\Omega = f\big(\hham(\vec{x})\big)dx \text{ – f. rozdziału / suma statystyczna}
 \end{gather*}
 
 \item obserwable po czasie
 \begin{gather*}
  \vec{A_t} = (t - t_o ) ^{-1} \int_{t_o}^t A(x(\tau)d\tau)
 \end{gather*}

 \item Hipoteza ergodyczna
\end{itemize}

\subsection{Metody dynamiki molekularnej}
\paragraph{} Metody dynamiki molekularnej służą do obliczania trajektorii w przestrzeni fazowej zbioru molekuł, z których każda podlega klasycznym równaniom ruchu.
\paragraph{} Komórka MD – komórka dynamiki molekularnej
\begin{gather} \label{r_1}
 \frac{du(t)}{dt} = K(u(t),t)
\end{gather}

\begin{gather*}
 \hham =  \sum_{i=1}^N \frac{p_i^2}{2m_i} + \sum_{K_j}U(u_{ij})
\end{gather*}

\subsubsection{Efekty wymiaru (finite size effects)}
\paragraph{} Problem skończonej wielkości obszaru. Rozwiązanie warunki brzegowe.

\begin{gather*}
 A(\vec{x}) = A(\vec{x} + \vec{n} \cdot L) \\
 \vec{n} = (n_1, n_2, n_3) \\
 L \text{ – liniowe wymiary układu}
\end{gather*}
\paragraph{} Przy obliczaniu energii całkowitej ze wzoru (\ref{r_1}) przyjmuje się tzw. konwencje najbliższego obrazu
\begin{gather}
 r_{ij} = \min_n \left\{ | \vec{r_i} - \vec{r_j} + \vec{n} L | \right\}
\end{gather}

cząstka oddziałuje tylko z N-1 cząstkami z komórki MD lub z jej najbliższego obszaru.


\subsection{Schematy całkowania}
MD jest zagadnieniem z warunkami początkowymi
\begin{gather}
 m \dot{r} = p_i \\
 \dot{p_i} = \sum_{i < j} F(x_j) \label{r_2}
\end{gather}

\paragraph{} (\ref{r_2}) wymaga $\frac{N(N-1)}{2}$ obliczeń jest to najbardziej złożona część obliczeniowo.

\paragraph{} Algorytm taki który działa krok po kroku. U podstaw metod numerycznych służących do rozwiązania równań różniczkowych stoi rozwinięcie w szereg Taylora. 

\begin{gather}
 u(t + h) = u(t) + \sum_{i}^{n - 1} \frac{h^i}{i!} u^{(u)}(t) + R_n 
 \end{gather}
 \begin{gather*}
 h \text{ – krok dyskretyzacji} \\
 R_n \text{ – reszta} 
\end{gather*}

Dla n = 2
\begin{gather}
 \frac{u}{t} = h^{-1} \left[ u ( t + h) - u (t) \right] + d(h) + o(h)
\end{gather}
\begin{gather}
 \frac{u}{t} = h^{-1} \left[ u ( t) - u (t - h) \right] + d(h) + o(h)
\end{gather}

Algorytm Eulera
\begin{gather}
 u(t+h) = u(t) + h K(u(t), t) \\
 \dot{u} = K(u(t),t)
\end{gather}

K - operator różniczkowy

Błąd - met. num.

Wybór kroku $h$ w zależności od przyjętej wielkości fluktuacji energii potencjalnej.

\subsection{obliczanie wielkości termodynamicznych}
\begin{gather*}
E = const. \\
T = const. \\
NVE
\end{gather*}
\paragraph{} średnia po trajektorii w takim układzie

\begin{gather*}
 \vec{A} = \langle A \rangle_{\text{NVE}} = \lim_{n \to \infty} (t' - t_o)^{-1} \int_{t_o}^{t'} A\big(\vec{r}^N(t), \vec{p}^N(t), V \big) dt\\
 \vec{E}_k = \lim_{\tau \to \infty} (t; - t_o)^{-1} \int_{t_o}^{t'} E_k(u(t))dt \\
 \vec{U} = \lim_{\tau \to \infty} (t; - t_o)^{-1} \int_{t_o}^{t'} U(V(t))dt
\end{gather*}

Korzystając z zasady ekwipartycji energii:
\begin{gather*}
 \frac{mv_i^2}{2} = \frac{k_BT}{2}
\end{gather*}

\begin{gather*}
 E_c = E_K + U
\end{gather*}

\subsection{Organizacja symulacji}
\begin{itemize}
 \item Inicjowanie – zadanie warunków początkowych – położenia i pędy
 \item Doprowadzenie do równowagi
 \item Realizacja według zadanego algorytmu
\end{itemize}

\begin{gather*}
 H = \frac{1}{2m} \sum_{i = 1}^{N} p_i^2 + \sum_{i < j} u(r_{ij})
\end{gather*}

z równania Newtona

\begin{gather*}
 \frac{d^2 V_i(t)}{dt^2} = \frac{1}{m}\sum_{i<j}F_i(r_{ij})
\end{gather*}

\subsection{Algorytm Verleta}
\paragraph{} Schemat rozwiązania równania Newtona jest następujący:

\begin{gather}
 \frac{d^2r_i}{dt^2} = \frac{1}{h^2} \Big[ r_i(t+h) - 2r_i(t) + r_i(t-h)\Big] = \frac{1}{m}F_i(t) \\ 
 r_i(t+h) = 2r_i (t) - r_i(t-h) + \frac{F_i(t)h^2}{m}
\end{gather}

\begin{gather*}
 t - h, t \to t + h \text{ - metoda 2kierunkowa}
\end{gather*}

\paragraph{} Stosując oznaczenia
\begin{gather*}
 t_n = nh \\
 r_i^n = r_i(t_n) \\
 F_i^n = F_i (t_n) \
\end{gather*}

\begin{gather} \label{r_3}
 r_i^{n+1} = 2r_i^n - r_i^{n-1} + \frac{F_i^n h^2}{m}
\end{gather}

\begin{gather} \label{r_4}
V_i^n = \frac{(r_i^{n+1} - r_i{n-1})}{2h}
\end{gather}

\subsection{Algorytm Verleta NVE ma postać:}
MD z alg Verleta
\begin{itemize}
 \item określenie położeń $r_i^o, r_i^1$
 \item określenie sił dla n tego kroku $F_i^n$
 \item określenie położeń dla $n+1$ kroku czasowego zw względu (\ref{r_3})
 \item obliczenie prędkości dla $n+1$-go kroku (\ref{r_4})
\end{itemize}

\subsection{Przykład układu monoatomowego z oddziaływaniem typu oddziaływania L-J}

\begin{gather*}
 U(r_{ij}) = 4\epsilon\left( \left(\frac{\sigma}{r_{ij}}\right)^{12} - \left(\frac{\sigma}{r_{ij}}\right)^6 \right) \\
 F_x = 48 \left( \frac{\epsilon}{\sigma^2} \right) (x_i - x_j) \left[\left(\frac{\sigma}{r_{ij}}\right)^{14} - \frac{1}{2} \left(\frac{\sigma}{r_{ij}}\right)^{8} \right]
\end{gather*}

\subsection{}
\begin{itemize}
 \item układ z N at. jednakowych (argonu)
 \item przeprowadzić symulacje MD w ukł $E_z$ = const.
 \item wylosować pędy z rozkładu maxwella
 \item wygenerować położenia w taki sposób aby atomy tworzyły komórkę elementarną kryształu gazu szlachetnego.
 \item histogramy pędów
 \item jmol - wizualizacja komórki
 \item E kin z rozkładu maxwella
 \item komórka elementarna(sześcian) jest w kuli at odbijają się od kuli
 \item przeskalowanie pędów tak by komórka/kryształ spoczywała $\vec{p_i}(0) = \vec{p_i}(0) = - \frac{1}{N}\sum_{i=1}{N}\vec{p_i}(0)$
\end{itemize}

potencjały i siły
\begin{itemize}
 \item Oddziaływanie van Der Walesa
\begin{gather*}
V^P = \epsilon (\frac{R}{r_{ij}}^{12} - 2 (R/r_{ij}) ^ 6) \\
r_{ij} = | \vec{r_i} - \vec{r_j}| \\ 
V^s = 0 \text{, gdy } r_i < L \\
V^s = \frac{1}{2} f ( r_i - L ) ^2 \text{, gdy } r_i \ge L
\end{gather*}

całkowity element potencjalny:
\begin{gather*}
V = \sum_{i=1}^{N-1}\sum_{j=0}^{i - 1} V^P(r_{ij}) + \sum_{i=1}^{N-1}V^? (r_{ij}) \\
\vec{F_i} = \sum_{0\le j \le N} F_{i(j)}^P + F_i^?
\end{gather*}

siły:
\begin{gather*}
\vec{F^p_{ij}} = \frac{dV^P(r_{ij})}{d\vec{V}} = 12 \epsilon (()^{12} - {}^6) \frac{\vec{r_i} - \vec{r_j}}{r_{ij}^2} \\
\vec{F^S} = - \frac{dV^s(x)}{d\vec{r_i}} = 0 r_i < L \\
\vec{F^S} = - \frac{dV^s(x)}{d\vec{r_i}} = f(L-r_i)\frac{\vec{r_i}}{r_i} r_i \ge L
\end{gather*}
\end{itemize}

\subsection{.xyz file}
n atomów np 216
nazwa
n razy:
  symbol atomu x\_1 y\_1 z\_1 np Ar 1 1 1

\subsection{}
\begin{itemize}
 \item C/C++
 \item double
 \item nie używać funkcji pow - patrz fastpow
 \item 
\end{itemize}


schemat numerycznych
$H = \frac{1}{2m} \sum_{i=1}^N   |p_i^2| + V$
równania ruchu postać:
\begin{gather*}
\dot{r}_i = dH/dp = p_i/m \\
\dot{p}_i = -dH/r_i = \vec{F_i}
\end{gather*}  
całkowanie z krokiem czasowym $\tau$
  t - dany krok czasowy symulacji
  
  $r_i$
\end{document}
